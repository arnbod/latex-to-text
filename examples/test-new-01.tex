
% Original (français)

$2+3$

$0+1$

$$4+5$$

\(6+7\)

\[8+9\]

\begin{equation}
2+2 = 2 \times 2
\end{equation}

\begin{activite}[Utiliser des fonctions]

\objectifs{Objectifs : utiliser des fonctions de \Python{} et du module \ci{math}.}

\begin{enumerate}
  \item  La fonction \Python{} pour le pgcd est \ci{gcd(a,b)}\index{gcd@\ci{gcd}}\index{pgcd} (sans le \og{}p\fg{}, pour \emph{greatest common divisor}). Calcule le pgcd de $a = 10\,403$ et $b = 10\,506$. Déduis-en le ppcm\index{ppcm} de $a$ et $b$. La fonction ppcm n'existe pas, tu dois utiliser la formule :
  $$\text{ppcm}(a,b) = \frac{a \times b}{\text{pgcd}(a,b)}.$$
  
  \item Trouve par tâtonnement un nombre réel $x$ qui vérifie toutes les conditions suivantes (plusieurs solutions sont possibles) :
  \begin{itemize}
    \item \ci{abs(x**2 - 15)} est inférieur à \ci{0.5}
    \item \ci{round(2*x)} renvoie \ci{8}
    \item \ci{floor(3*x)} renvoie \ci{11}
    \item \ci{ceil(4*x)} renvoie \ci{16} 
  \end{itemize}
 
  \emph{Indication.} \ci{abs()}\index{abs@\ci{abs}}\index{valeur absolue} désigne la fonction valeur absolue.
  
  \item Tu connais la formule de trigonométrie 
  $$\cos^2 \theta + \sin^2 \theta = 1.$$
  Vérifie que pour $\theta = \frac\pi7$ (ou d'autres valeurs) cette formule est numériquement vraie. (Ce n'est pas une preuve de la formule, car \Python{} ne fait que des calculs approchés du sinus et du cosinus).
\end{enumerate}  
\end{activite}
